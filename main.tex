% ----------------------------------------------------------------------
%                   LATEX TEMPLATE FOR PhD THESIS
% ----------------------------------------------------------------------

% based on Harish Bhanderi's PhD/MPhil template, then Uni Cambridge
% http://www-h.eng.cam.ac.uk/help/tpl/textprocessing/ThesisStyle/
% corrected and extended in 2007 by Jakob Suckale, then MPI-CBG PhD programme
% and made available through OpenWetWare.org - the free biology wiki


%: Style file for Latex
% Most style definitions are in the external file PhDthesisPSnPDF.
% In this template package, it can be found in ./Latex/Classes/
\documentclass[twoside,11pt]{PhDthesisPSnPDF}


%: Macro file for Latex
% Macros help you summarise frequently repeated Latex commands.
% Here, they are placed in an external file /Latex/Macros/MacroFile1.tex
% An macro that you may use frequently is the figuremacro (see introduction.tex)
% \include{Latex/Macros/MacroFile1}
\usepackage[T1]{fontenc}
\usepackage{array}
\usepackage{pdfpages}
\usepackage{xspace}
\usepackage{xcolor}
\usepackage{lipsum}
\usepackage{enumerate}
\usepackage[shortlabels]{enumitem}
\newlist{todolist}{itemize}{2}
\setlist[todolist]{label=$\square$}

\usepackage[capitalise,noabbrev]{cleveref}
\usepackage{booktabs}
\usepackage{tikz}
\usepackage{caption}
\usepackage{subcaption}
\PassOptionsToPackage{hyphens}{url}
\usepackage{hyperref}
\usepackage{minted}
\usepackage{listings}
\usepackage{array,ragged2e}
\newcolumntype{P}[1]{>{\RaggedRight\arraybackslash}p{#1}}

%\usepackage{graphics}
% or use the graphicx package for more complicated commands
%\usepackage{graphicx}

%: ----------------------------------------------------------------------
%:                  TITLE PAGE: name, degree,..
% ----------------------------------------------------------------------
\usepackage{graphicx}

      \textwidth 15cm
      \textheight 22cm
      \parindent 10pt
      \oddsidemargin 0.85cm
      \evensidemargin 0.37cm
     
\newcommand{\ie}{\emph{i.e.,}\xspace}
\newcommand{\eg}{\emph{e.g.,}\xspace}
\newcommand{\etc}{etc.\xspace}
\newcommand{\etal}{\emph{et~al.}\xspace} 

\newcommand{\todo}[1]{\textcolor{blue}{#1}} 

\newcommand{\pangraph}[1]{\vspace{-13pt}\paragraph{#1}}
\newcommand{\Sec}[1]{Section~{\ref{#1}}}
\newcommand{\chap}[1]{Chapter~{\ref{#1}}}
\newcommand{\fig}[1]{Figure~\ref{#1}}
\newcommand{\tab}[1]{Table~\ref{#1}}

\newcommand{\Plus}{\texttt{+}} 

\begin{document}

% animesh - remove these when preparing the final thesis version 
\chapter*{Notes about this document}
\section*{Revision history}
\begin{itemize}
    \item v2.0     : Updated details on each chapter, and a new thesis checklist.
    \item v1.5     : Added references and tex template link to the Reproducibility Appendix. 
    \item pre-v1.4 : Initial versions. 
\end{itemize}

\section*{What is this about}
Over the past years I have given a common subset of advice regarding how to write your thesis to multiple BSc 
and MSc students. This document is a synthesis of these advises, dos and don'ts that I have told them. 
In the coming section, I will add details of the kind of content you should be covering in your thesis and 
what I expect to see in each section of your thesis. BSc and MSc theses differ in the scope of the problem that 
they cover. Also, these are a broader set of advice that will also help when you are writing a research paper. 


I am happy to have input from you in this matter and let me know if I have missed something that you covered in 
your thesis, and/or how we can improve the instructions here. This is not meant to be a comprehensive set 
of instructions, and following these guidelines does not imply that you will be awarded the maximum grade. 

\section*{General Guidelines}
\begin{itemize}
    \item Aim to produce a first complete draft 2-3 weeks before the final hand-in deadline. The earlier you give me a draft to review, more detailed feedback you will get. 
    \item Spell check and proofread your draft before you give me. Do not leave half broken sentences and empty subsections.
    \item Aim to write frequently, and ask questions early. 
    \item You will have one major and one minor revision opportunity before you hand in your final thesis draft for grading. It is unreasonable for me to read and re-read your drafts every week, hence, aim to write a first as complete draft as possible for maximum feedback. 
    \item When writing \textbf{\textit{"Always think like a compiler - have I explained and/or defined all the ideas that I am about to use?"}}. It is a responsibility of the writer to write an easy to read document, not the other way around. Do not just write for the writing sake to fill up pages. Try to make a simple, linear structure of your thesis. Build your case step by step, one new idea at a time, one new configuration or complexity. Try to reduce the "cognitive load" of your thesis as much as possible. If you want to test what does \textit{"cognitive load"} mean, then try to read your own thesis very late in the night when you are really tired or fresh in the morning when you do not remember all the details up front. Re-read what you have written after a couple of days to see if you remember what is the key message that you were trying to get across. 
    \item Always read and re-read your own thesis before you submit for a review. 
    \item Your thesis PDF is for everyone else than you and me. Think of your friend who also studies Computer Science but does not know what you work on. They are not invested in your problem, and are not talking to you over the last couple of months. Hence, think and write down all the details which are necessary for them to understand what you have done. 
\end{itemize}

\subsection*{English Usage}
Pay attention to your English usage, especially the use of a paragraph and subsection. Each paragraph is supposed to cover one logical idea. Do not ramble. Do not cover 100 different possibilities and ideas in a single paragraph. After each paragraph, consider does it explain one idea and one idea only cleanly. Do not write paragraphs which are 1 or 2 pages. For every paragraph connect it back and forward. Similar advice goes for subsections. Think, if the heading of the subsection makes sense. If I hide the title of your subsection, would a reader come up with a similar heading if she or he reads the content of your subsection. For every new idea: write what you are about to explain, explain it, and recap what you have explained. 


Do not use colloquial terms in your writing. Read papers and see how they present information and ideas. Do not use terms like "may", "could", "perhaps" - try to write as cleanly as possible without ambiguity. Always write in the present tense in active voice. Do not use frivolous adjectives and/or adverbs like "very much", "tremendous", "very fast", "high overheads". Always aim to quantify. What does "fast" or  "slow" mean, can you measure it? in what term, what context? Be as specific as possible when you report information. 

\textbf{The golden rule of writing} is (follow this for each subsection, section, and chapter) (i) Write what you are going to tell the reader and why is it important; (ii) Write and explain it to the reader; (iii) Recap what you have just explained to the reader.  

\subsection*{Mandatory Reading List}
Please read the following references carefully as they cover many of ideas which are iterated throughout this document:  
\begin{enumerate}
    \item George D. Gopen \& Judith A. Swan, The Science of Scientific Writing, \url{https://github.com/animeshtrivedi/notes/blob/master/docs/the-science-of-scientific-writing.pdf}. 
    \item Gernot Heiser, Tips and Guidance for Students Writing Papers and Reports, \url{https://www.cse.unsw.edu.au/~gernot/style-guide.html}. 
    \item Gernot Heiser, Systems Benchmarking Crimes, \url{https://www.cse.unsw.edu.au/~gernot/benchmarking-crimes.html}. 
    \item John Ousterhout, Always Measure One Level Deeper, \url{https://m-cacm.acm.org/magazines/2018/7/229031-always-measure-one-level-deeper/fulltext}. 
    \item Roy Levin, and David D. Redell,  How (and how not) to write a good systems paper, (applicable to your thesis work as well), \url{https://www.usenix.org/conferences/author-resources/how-and-how-not-write-good-systems-paper}.
    \item Inclusion and Diversity in Writing, \url{https://acmsocc.github.io/2020/inclusion_and_diversity_in_writing.html}. 
\end{enumerate}

\subsection*{Where can I see examples of past theses}


\subsection*{In the End}
Take time to develop writing skills. It is going to be good from one day to another. It is as important as learning how to code, if not more. Write drafts and seek feedback. You can be creative in your writing once you have mastered the basics. 

Most importantly, your thesis work is probably the biggest piece of work you have done so far in your studies. Care for it. Take pride in your work and writing - your name is attached to it! 


\newpage 
\section*{Thesis checklist}
  \begin{todolist}
    \item Thesis statement
    \item List item 2 goes here.
    \begin{todolist}
      \item Sublist item 1 goes here.
      \item Sublist item 2 goes here.
    \end{todolist}
    \item List item 3 goes here
    \item List item 4 goes here.
  \end{todolist}


\thispagestyle{empty}

\begin{center}

\vspace*{-3.5cm}

Vrije Universiteit Amsterdam \hspace*{2cm} Universiteit van Amsterdam

\vspace{1mm}

\hspace*{-6.5cm}\includegraphics[height=20mm]{vu-griffioen.pdf}

\vspace*{-2cm}\hspace*{7.5cm}\includegraphics[height=15mm]{uva_logo.jpg}

\vspace*{1.3cm}
%\includegraphics[height=45mm]{figs/DPFS-logo.pdf}
\vspace*{4.5cm}

{\Large Master Thesis}

\vspace*{0.5cm}

\rule{.9\linewidth}{.6pt}\\[0.4cm]
{\huge \bfseries Thesis title  \par}\vspace{0.4cm}
{\large \bfseries Animesh's notes on CS BSc and MSc Thesis \par}\vspace{0.4cm}


\rule{.9\linewidth}{.6pt}\\[1.5cm]

\vspace*{2mm}

{\Large
\begin{tabular}{l}
{\bf Author:} ~~Student name ~~~~ (student number)
\end{tabular}
}

\vspace*{2cm}

\begin{tabular}{ll}
{\it 1st supervisor:}   & ~~ Prof. \ldots ~~~~~~~~~ Vrije Universiteit Amsterdam \\
{\it daily supervisor:} & ~~ Prof. \ldots ~~~~~~~~~ ??? \\
{\it 2nd reader:}       & ~~ Prof. \ldots ~~~~~~~~~ Vrije Universiteit Amsterdam
\end{tabular}

\vspace*{2.5cm}

\textit{A thesis submitted in fulfillment of the requirements for\\ the joint UvA-VU Master of Science degree in Computer Science}

\vspace*{1.8cm}

\today\\[4cm] % Date

\end{center}

\newpage


% ----------------------------------------------------------------------
       
% turn of those nasty overfull and underfull hboxes
\hbadness=10000
\hfuzz=50pt


%: --------------------------------------------------------------
%:                  FRONT MATTER: dedications, abstract,..
% --------------------------------------------------------------


%\language{english}


% sets line spacing
\renewcommand\baselinestretch{1.2}
\baselineskip=18pt plus1pt


%: ----------------------- generate cover page ------------------------

\begin{center}
\textit{``There are two hard things in computer science: cache invalidation, naming things, and off-by-one errors.''}  - Jeff Atwood \\ \url{https://twitter.com/codinghorror/status/506010907021828096?lang=en}
\end{center}

\newpage


%: ----------------------- abstract ------------------------

% Your institution may have specific regulations if you need an abstract and where it is to be placed in the document. The default here is just after title.

%\include{0_frontmatter/abstract}

\begin{abstracts} 
An abstract is a compressed or zip summary of your thesis. You can follow a simple structure and try to fill in the following sequence of details:
\begin{itemize}[leftmargin=10pt,itemindent=0em,nolistsep]
  \item what is the broader societal, economic, scientific context in which this work is done, who are the people who might be interested in your work (2-4 sentences);
  \item what is changing (or why now?) and what broad problem(s) does this change creates, has the problem always been here or something has triggered this problem, 
  is it a specific problem or part of a bigger trend in your field (2-4 sentences);
  \item what specific scientific/research problems that this thesis focuses on (4-6 sentences). Please state as a clear statement: ``The key scientific question that this thesis addresses is \ldots'' ;
  \item how does this thesis work tackle the problem - survey, design, implementation, evaluation (4-6 sentences);
  \item what are the key findings, give "quantitative results". For example, \textit{Our results show that our system decreases SQL query execution time by 50.6\% in the cloud} (1-2 sentence);
  \item how can I use this thesis work? Please always add a line stating, ``The code (or data) for this thesis work is openly available at \url{https:// . . . } ''.   
\end{itemize}

These parts are guidelines, not strict rules (except the last one). Based on the flavor of the thesis you do, you will budget your 
word/sentence quotas differently. In your writing be precise, and be specific about the work which is done in this thesis as much as possible. 

Try to keep the abstract within a single page. 
\end{abstracts}

% The original template provides and abstractseparate environment, if your institution requires them to be separate. I think it's easier to print the abstract from the complete thesis by restricting printing to the relevant page.
% \begin{abstractseparate}
%   \input{Abstract/abstract}
% \end{abstractseparate}


%: ----------------------- tie in front matter ------------------------

\frontmatter
%\include{0_frontmatter/dedication}
%\include{0_frontmatter/acknowledgement}


%: ----------------------- contents ------------------------

\setcounter{secnumdepth}{2} % organisational level that receives a numbers
\setcounter{tocdepth}{1}    % print table of contents for level 3
\tableofcontents            % print the table of contents
% levels are: 0 - chapter, 1 - section, 2 - subsection, 3 - subsection


%: ----------------------- list of figures/tables ------------------------

\listoffigures	% print list of figures

\listoftables  % print list of tables




%: ----------------------- glossary ------------------------

% Tie in external source file for definitions: /0_frontmatter/glossary.tex
% Glossary entries can also be defined in the main text. See glossary.tex
% 
%\include{0_frontmatter/glossary} 

%\begin{multicols}{2} % \begin{multicols}{#columns}[header text][space]
%\begin{footnotesize} % scriptsize(7) < footnotesize(8) < small (9) < normal (10)

%\printnomenclature[1.5cm] % [] = distance between entry and description
%\label{nom} % target name for links to glossary

%\end{footnotesize}
%\end{multicols}




%: --------------------------------------------------------------
%:                  MAIN DOCUMENT SECTION
% --------------------------------------------------------------

% the main text starts here with the introduction, 1st chapter,...
\mainmatter

\renewcommand{\chaptername}{} % uncomment to print only "1" not "Chapter 1"


%: ----------------------- subdocuments ------------------------

% Parts of the thesis are included below. Rename the files as required.
% But take care that the paths match. You can also change the order of appearance by moving the include commands.

\chapter{Introduction}\label{s:introduction}
\chapter{Background}\label{s:background}      
            
% --------------------------------------------------------------
%:                  BACK MATTER: appendices, refs,..
% --------------------------------------------------------------

% the back matter: appendix and references close the thesis


%: ----------------------- bibliography ------------------------

% The section below defines how references are listed and formatted
% The default below is 2 columns, small font, complete author names.
% Entries are also linked back to the page number in the text and to external URL if provided in the BibTex file.

% PhDbiblio-url2 = names small caps, title bold & hyperlinked, link to page 
%\begin{multicols}{2} % \begin{multicols}{ # columns}[ header text][ space]
%\begin{tiny} % tiny(5) < scriptsize(7) < footnotesize(8) < small (9)

%\bibliographystyle{Latex/Classes/PhDbiblio-url} % Title is link if provided
\renewcommand{\bibname}{References} % changes the header; default: Bibliography

%\bibliography{references} % adjust this to fit your BibTex file
\bibliography{main}

%\include{sections/appendix}

%\end{tiny}
%\end{multicols}



% --------------------------------------------------------------
% Various bibliography styles exit. Replace above style as desired.

% in-text refs: (1) (1; 2)
% ref list: alphabetical; author(s) in small caps; initials last name; page(s)
%\bibliographystyle{Latex/Classes/PhDbiblio-case} % title forced lower case
%\bibliographystyle{Latex/Classes/PhDbiblio-bold} % title as in bibtex but bold
%\bibliographystyle{Latex/Classes/PhDbiblio-url} % bold + www link if provided

%\bibliographystyle{Latex/Classes/jmb} % calls style file jmb.bst
% in-text refs: author (year) without brackets
% ref list: alphabetical; author(s) in normal font; last name, initials; page(s)

%\bibliographystyle{plainnat} % calls style file plainnat.bst
% in-text refs: author (year) without brackets
% (this works with package natbib)


% --------------------------------------------------------------

% according to Dresden med fac summary has to be at the end
%\include{0_frontmatter/abstract}

%: Declaration of originality
%\include{8_backmatter/declaration}


\end{document}

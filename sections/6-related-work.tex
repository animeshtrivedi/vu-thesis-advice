\section{Related Work}
It is often a misconception that you need to defend against the related work. Related work is your friend. Show your evaluator that you are aware of the state of the art in your field of research. Think critically, which small design choices and pieces remind you of which paper. What is your original contribution and what you can claim from others. More comprehensive your related work is, the more I will believe your contributions. 

\subsection*{Where can I find related work?}
At the start of your thesis I will give you an initial set of papers and conference names. Read those papers. Pay attention to their related work, and read those. Synthesize the common keywords and terminology that these papers use, and then search those on Google Scholar, or any other academic research engine. See which conferences those papers are published in, and then go check previous iterations of those conferences. Pay attention to the research groups and universities who write those papers, and then go check their other works on their homepage. Even after all of this you cannot find any other paper, come talk to me, but then you have to show me what you have done so far. 

\subsection*{How many references are enough?}
There is no right answer here. For a well established filed, you might be looking papers in 50-100s. For a new field, there might only be a handful of papers. Try to identify the impactful papers and try to include them. However, in any circumstances including one or two papers is never good. In that case try to expand what constitutes related work and show that you are aware of a new established field and other nearby areas from which the new field gets ideas. You can be creative with this. For example if you do not find enough related work on programmable storage, then you can include programmable network, languages, compiler work in your related work. 


\subsection*{What happens if I miss a paper?}
No need to panic. In broad and fast moving research areas of Computer Science, there will be non-zero probability that we might have missed a paper. Often it is a simple process just to add the missed paper in the discussion without much restructuring. Though it is always preferred to avoid such situations, especially if the missed paper is really an important one. 

For more information regarding how to do a survey in a new filed, please see my advice on how to do a literature study \url{https://animeshtrivedi.github.io/lit-study/}. 

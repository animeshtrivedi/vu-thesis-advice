\chapter{Background on \ldots}\label{s:background} 
Start each section or chapter on a new page. 

Provide necessary background information, concepts, and ideas which are necessary to understand your thesis work. Use examples. Use figures. Use code snippets. Take your time and explain things. Often when we spend too much time with a system we forget what we did not know when we started the work. Keep track of your activities during your thesis work and everything that you had to learn, a new language, a new framework, a new compiler, a new software package design, a new concept all should be explained here cleanly. 


How detailed your information should be? Whatever is necessary to understand your contribution. Often a single topic can be really large. For example, if you talk about distributed transactions then the background knowledge can be about providing ACID guarantees, or performance, or concurrency management, or membership management, or consensus, etc. Based on what topic your thesis is tacking, you should only focus on selectively introducing those ideas. You can leave a broader reference saying that \textit{More details about the distributed transaction design can be found in the survey of CoolPeople et al. from ACM CSUR 2020 [x]}. While explaining technical concepts you can slowly prepare the reader about the next chapter and what is about to come. 

Take your time to think in which order you should be presenting background information. Do not make a laundry list of random topics. Write them coherently. 

At the end of your background chapter, a knowledgeable person or expert in your field should understand your problem completely. It is always better to spell it out clearly. For example, \textit{So far in this chapter we have presented background information about how serverless and container technology work, what are the big container repositories. In Section X.Y we have further elaborated challenges associated with the scalability, complexity and deployment. In the next section, we will present MyCoolSystem that proposes combining light-weight language virtualization with Lambda to deliver high performance, cold booting of Lambdas.}

% animesh - remove these when preparing the final thesis version 
\chapter*{Notes about this document}
\section*{Revision history}
\begin{itemize}
    \item v2.0     : Updated details on each chapter, and a new thesis checklist.
    \item v1.5     : Added references and tex template link to the Reproducibility Appendix. 
    \item pre-v1.4 : Initial versions. 
\end{itemize}

\section*{What is this about}
Over the past years I have given a common subset of advice regarding how to write your thesis to multiple BSc 
and MSc students. This document is a synthesis of these advises, dos and don'ts that I have told them. 
In the coming section, I will add details of the kind of content you should be covering in your thesis and 
what I expect to see in each section of your thesis. BSc and MSc theses differ in the scope of the problem that 
they cover. Also, these are a broader set of advice that will also help when you are writing a research paper. 


I am happy to have input from you in this matter and let me know if I have missed something that you covered in 
your thesis, and/or how we can improve the instructions here. This is not meant to be a comprehensive set 
of instructions, and following these guidelines does not imply that you will be awarded the maximum grade. 

\section*{General Guidelines}
\begin{itemize}
    \item Aim to write frequently, and ask structural questions early.
    
    \item Aim to produce a first complete draft \underline{4 weeks before the final hand-in deadline}. The earlier you 
    give me a draft to review, more detailed feedback you will get. My reviews are as comprehensive as your writing. 
    If you give me a really rough draft to review, my feedback will be incomplete. At this stage of writing, your introduction, 
    scope of the thesis work, and research questions should be clearly established.   
    
    \item You will have one major and one minor revision opportunity before you hand in your final thesis draft for grading. 
    It is unreasonable for me to read and re-read your drafts every week, hence, aim to write a first as complete 
    draft as possible for maximum feedback.
     
    \item When writing \textbf{\textit{"Always think like a compiler - have I explained and/or defined all the ideas that I am about to use?"}}. 
    It is a responsibility of the writer to write an easy to read document, not the other way around. Do not just 
    write for the writing sake to fill up pages. Try to make a simple, linear structure of your thesis. 
    Build your case step by step, one new idea at a time, one new configuration or complexity. Try 
    to reduce the "cognitive load" of your thesis as much as possible. If you want to test what 
    does \textit{"cognitive load"} mean, then try to read your own thesis very late in the night when you are really 
    tired or fresh in the morning when you do not remember all the details up front. Re-read what you have written 
    after a couple of days to see if you remember what is the key message that you were trying to get across.
    Good writing takes time.  
    
    \item Always read and re-read your own thesis before you submit for a review. 
    
    \item Your thesis PDF is for everyone else than you and me. Think of your friend who also studies Computer Science 
    but does not know what you work on. They are not invested in your problem, and are not talking to you over the 
    last couple of months. Hence, think and write down all the details which are necessary for them to understand 
    what you have done.
    
    \item Spell check and proofread your draft before you give me. Do not leave half broken sentences and empty 
    subsections.
     
\end{itemize}

\subsection*{English Usage}
Pay attention to your English usage, especially the use of a paragraph and subsection. Each paragraph is supposed 
to cover one logical idea (and one idea only!). Do not ramble. Do not cover 100 different possibilities and ideas 
in a single paragraph. After each paragraph, consider does it explain one idea and one idea only cleanly. Do not 
write paragraphs which are 1 or 2 pages. For every paragraph connect it back and forward. A similar advice goes for 
section, and subsections. How all of them connect to each other? Think, if the heading of the subsection makes sense. 
If I hide the title of your subsection, would a reader come up with a similar heading if she or he reads the content 
of your subsection. For every new idea: write what you are about to explain, explain it, and recap what you have 
explained. 


Do not use colloquial terms in your writing. Read papers and see how they present information and ideas. Do not 
use terms like "may", "could", "perhaps" - try to write as cleanly as possible without ambiguity. 
Always write in the present tense in active voice. Do not use frivolous adjectives and/or adverbs like 
"very much", "tremendous", "very fast", "high overheads". Always aim to quantify. What does "fast" or  "slow" mean, 
can you measure it? in what term, what context? Be as specific as possible when you report information. 


\textbf{The golden rule of writing} is (follow this for each subsection, section, and chapter) 
(i) Write what you are going to tell the reader and why is it important; 
(ii) Write and explain it to the reader; 
(iii) Recap what you have just explained to the reader.  


\subsection*{Mandatory Reading List}
Please read the following references carefully as they cover many of ideas which are iterated throughout this document:  
\begin{enumerate}
    \item George D. Gopen \& Judith A. Swan, The Science of Scientific Writing, \url{https://github.com/animeshtrivedi/notes/blob/master/docs/the-science-of-scientific-writing.pdf}. 
    \item Gernot Heiser, Tips and Guidance for Students Writing Papers and Reports, \url{https://www.cse.unsw.edu.au/~gernot/style-guide.html}. 
    \item Gernot Heiser, Systems Benchmarking Crimes, \url{https://www.cse.unsw.edu.au/~gernot/benchmarking-crimes.html}. 
    \item John Ousterhout, Always Measure One Level Deeper, \url{https://m-cacm.acm.org/magazines/2018/7/229031-always-measure-one-level-deeper/fulltext}. 
    \item Roy Levin, and David D. Redell,  How (and how not) to write a good systems paper, (applicable to your thesis work as well), \url{https://www.usenix.org/conferences/author-resources/how-and-how-not-write-good-systems-paper}.
    \item Inclusion and Diversity in Writing, \url{https://acmsocc.github.io/2020/inclusion_and_diversity_in_writing.html}. 
\end{enumerate}


\subsection*{Examples of Past Theses}

All students who have worked with me have their thesis, reports, surveys publicly available on: \url{https://animeshtrivedi.github.io/team/}


There are more advice and resources available at our group's website: \url{https://atlarge-research.com/new_students.html}


\subsection*{In the End}
Take time to develop writing skills. It is not going to get better from one day to another. It is as important as 
learning how to code, if not more. Write drafts and seek feedback. You can be creative in your writing once you 
have mastered the basics. 


Most importantly, your thesis work is probably the biggest piece of work you have done so far in your studies. 
Care for it. Take pride in your work and writing - \textit{your name is attached to it!} 


\newpage 
\section*{Thesis checklist}
  \begin{todolist}
    \item Names of the supervisors and 2nd reader included on the front page?   
    \item Is the table of content max 2-level deep only (no subsection 2.3.1)? Please keep it at 2.3. 
    \item Are chapter, section, subsection heading descriptive?
    \item Does the abstract clearly contain (i) scientific questions; (ii) thesis contributions; and (iii) link to the open-source code/artifact?
    \item Is there a thesis statement? 
    \item List item 2 goes here.
    \begin{todolist}
      \item Sublist item 1 goes here.
      \item Sublist item 2 goes here.
    \end{todolist}
    \item List item 3 goes here
    \item List item 4 goes here.
  \end{todolist}
